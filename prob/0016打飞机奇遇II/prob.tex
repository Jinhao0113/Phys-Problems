\documentclass{article}
\usepackage{ctex}
\usepackage{amsmath}
\usepackage{graphicx}
\usepackage{wrapfig}
\usepackage{caption}
\usepackage[top=0.8in, bottom=0.8in,left=0.8in, right=0.8in]{geometry}
\usepackage{float} 
\usepackage{subfigure}
\usepackage{subcaption}
\usepackage{bm}
\xeCJKsetup{CJKmath=true} 

\begin{document}
\section*{打飞机奇遇II(40分)}
经过巨长时间,小s同学终于早好了电磁炮,准备开始打飞机了。
\begin{itemize}
\item[(1)]小s同学想先检验自己的力量,于是先用手扔炮弹,已知此速度下空气阻力为$\vec{f}=-k\vec{v}=-m\beta\vec{v}$,小s同学抛出炮弹的速度为$v_0$,与竖着方向夹角为$theta_0$,质量为m,重力加速度为g,以抛出点为远点,求$x(t),y(t)$
\item[(2)]没上过几节体育课的小s同学力量不够,炮弹达不到飞机,于是他启用了新建的电磁炮,在此速度下阻力近似为$\vec{f}=-m\beta|\vec{v}|\vec{v}=-c|\vec{v}|\vec{v}$,已知初速度为$v_0$,\textbf{角度为$\theta_0$(与上题不同,此为与水平方向夹角)},重力加速度为$g$,质量为$m$。
\begin{itemize}
\item[(2.1)]列出自然坐标系下的动力学方程(可带曲率半径$\rho$)
\item[(2.2)]以$v,\theta$为变量列出微分方程并积分的$v(\theta)$,并得到炮弹最高点的速度,再代入 $\beta =0$检验你的结果.\par
提示:用$\rho$的自然坐标表示式,并用$v\cos\theta$换元.\par
    \[
    \int \dfrac{\mathrm{d}\theta}{\cos^3\theta}=\dfrac{1}{2}\left(\dfrac{\sin\theta}{\cos^2\theta}+\ln\dfrac{1+\sin\theta}{\cos\theta}\right)
    \]
\item[(2.3)]若用电磁炮直接轰击漂亮国的首都,(不考虑地球弯曲),且出射速度极大$\theta_0=0$轨道近似为直线,在此条件下求解$x$关于$t$的函数,并求出轨迹方程$y(x)$ (提示:用$\rho$的直角坐标表示)
\end{itemize}
\end{itemize}
\end{document} 