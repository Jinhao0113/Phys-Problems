\documentclass{article}
\usepackage{ctex}
\usepackage{amsmath}
\usepackage{graphicx}
\usepackage{wrapfig}
\usepackage{caption}
\usepackage[top=0.8in, bottom=0.8in,left=0.8in, right=0.8in]{geometry}
\usepackage{float} 
\usepackage{subfigure}
\usepackage{subcaption}
\usepackage{bm}
\xeCJKsetup{CJKmath=true} 

\begin{document}
\section*{可控核聚变约束装置}
可控核聚变一直是人类的梦想,最早关于核聚变的研究可追溯到二战之后不久。但是如何约束超高温的等离子体是一个非常重要的问题在这里我们来研究其中的两种$\Theta$ 约束与$\mathrm{Z}$约束.
\subsection*{等离子体物理相关}
等离子体是由阳离子、中性粒子、自由电子等多种不同性质的粒子所组成的电中性物质,其中阴离子(自由电子)和阳离子分别的电荷量相等。在极高温下,电子完全脱离原子核束缚,故其可看作等离子体。
\begin{itemize}
    \item[(1)]假设正离子或电子的数密度为$n$。假设正离子只电离出一个电子,现在坐标原点出放置一个点电荷$Q$,求其电势分布.
    \item[(2)] 
\end{itemize}
\end{document} 